\begin{acknowledgment}
本研究を行うにあたり、非常に多くの方々にお力添えをいただきましたこと、深く感謝致します。特に指導教官である片岡先生は、私の希望に沿って「陽子線CT」という医療物理の最先端を行く大変興味深い研究テーマを与えてくださり、いつも熱心かつ懇親丁寧なご指導を賜りました。また必要な実験を問題なく行えるよう、ビームタイムを獲得するためにご尽力いただきました。片岡先生に初めてお会いしたのは学部1年生の応用物理学ゼミナールでしたが、大学に入ったばかりで専門知識が全くない私に、物理過程や実験方法を噛み砕いて分かりやすく教えてくださり、また「研究」のイメージをもつきっかけを与えてくださりました。そのときから、片岡先生の元で研究を行うことを強く望んでおりました。学部4年生になり研究室に配属されてからも、困難にぶつかったときはご助言いただき、いつも新たな視点で考えることができました。夏には学部生でありながら国際学会に参加するという大変貴重な経験をすることが出来ました。研究室で過ごしたこの1年間は新しいことの連続でした。このような経験をすることが出来たのも片岡先生のおかげであり、要領が悪く失敗することの多い私を最後まで見守ってくださったことを心から御礼を申し上げたいと思います。今後も多くのご迷惑をおかけするかと思いますが、一生懸命研究に打ち込みますので、ご指導ご鞭撻の程どうぞよろしくお願い致します。\\
 広大の西尾先生、東大の田中さんは、本研究を行うと決まった当初から、実験で使用している装置や実験方法に始まり、失敗談や気をつけるポイントなどを親切に教えてくださいました。また、外部実験に付き添っていただきその場で様々なご助言いただきました。一年間で再構成画像の取得に成功したのは、西尾先生や田中さんのおかげであり、大変感謝しております。\\
 放医研の稲庭先生には、いつもビームタイムの調整をしていただきました。実験資料回覧時には的確かつ重要なご助言をいただき、これにより自分の研究の間違いに気づき修正出来ました。\\
 名古屋陽子線治療センターの歳藤先生は、長時間の実験にも関わらずお付き添いくださり、効率的かつ正確な実験が行えるように多くのアイディアをいただきました。また多重クーロン散乱の補正に関して、医学的な観点から方法をご提案いただきました。\\
 杉浦さんや北村さん、オペレータの皆様には、私たちが難なく思い通りに実験を行えるよう、必要な実験装置を提供していただきました。危機管理が甘い私たちのことを心配して、実験に付き添っていただいたこともありました。\\
 研究室の先輩方にも、多くのご指導を頂戴致しました。\\
 岸本さんには、研究室配属当初から大変お世話になりました。実験機器の使い方や解析の仕方などを初めとする実験のやり方について、一から丁寧に教えていただきました。また、思うような成果が出せず実験の方向性に悩んだときに、親身になって相談に乗ってくださいました。研究者としてだけではなく一人の女性としても、大変尊敬する先輩です。\\
 辻川さんには、本研究の最も大きな課題である「多重クーロン散乱」の補正方法に関して、多くのご意見をいただき、また参考になる論文をご紹介いただきました。修論執筆でお忙しい中何度も、夜遅くまで残って研究についてアドバイスをくださりました。特に、私の研究においてご自身の研究テーマと似た部分に関しては、私が知らない多くの知識を御教授いただきました。マイペースで行動に移るまでに時間がかかる私のことを心配して、毎日研究の進捗等に関して話しかけてくださり、研究以外の悩みについても時間を割いて相談に乗ってくださいました。卒業されてしまうことは大変悲しいですが、今後のご活躍をお祈り致します。\\
 多屋さんには、外部実験に付き添っていただき、実験中に出た疑問点について一緒に考えていただきました。陽子線のプロフェッショナルであり、研究を進めるにあたり分からないことが出てきたときは、いつも分かりやすく教えてくださいました。プログラミングの分からない私に基礎から丁寧に教えてくださり、飲み込みが悪く同じミスをすることも多々ありましたが、その度に指摘をしてくださいました。多屋さんの話しかけやすい人柄から頼ることが多かったですが、いつでもご親切に対応していただきました。\\
 大島さんは「CT」を扱う点で研究内容が似ており、画像再構成を行う際に参考書をご紹介いただき、マクロの内容を基礎から噛み砕いて教えてくださいました。卒業研究のまとめ方に悩んでいたときには、流れを一緒に考えてくださり、実験についてのご助言もいただきました。卒論発表の前にはスライドの流れから発表の仕方にいたるまで、ご丁寧にご指導いただきました。心配性であり何かと不安になることの多い私を気にかけ、いつでも相談に乗ってくださり、精神面でも支えていただきました。大島さんは私が憧れる先輩です。\\
 岩本さんはパソコンに関して大変詳しく、多くの知識を御教授くださいました。またムードメーカーであり、いつも研究室を明るくしてくださいました。一年間楽しく研究を行えたのは、岩本さんがいらっしゃったからです。\\
 三村さんにも、プログラミングに困ったときに大変お世話になりました。休日にも研究をされており、研究に対する姿勢を見習いたいと思います。\\
 同期の増田君は、毎回外部実験に付き添い実験を効率よく進められるように手伝ってくれました。また研究において行き詰まったときには、持ち前の多くの知識で様々な視点から考察を行い、一緒に解決策を見つけようと努力してくれました。いつでも嫌な顔せず親身になって問題解決の糸口を探してくれて、本当にありがとうございました。今後はより自立して研究を行い、私も周りに還元出来るように精進したいと考えております。森田君にも、実験についてきてもらっただけでなく、陽子線との違いを理解するためにX線に関して多くのことを教えてもらいました。さらに、秋田君、小出さん、末岡君、藤原さんという優秀な同期に恵まれ、お互いに叱咤激励して高め合うことが出来ました。\\
 最後になりましたが、いつでも全力で私のことを応援してくれて、資金面だけではなく生活面においても多くの支援をしてくれた両親に、また私の悩みを理解し自分の経験をもとにいつも様々なアドバイスをくれた姉に、心から深く感謝しております。心配や迷惑をかけてばかりですが、家族の支えがあって今があると強く感じております。\\
 私がこうして卒業研究を行い論文を執筆できたことは皆様の支えがあったからこそであります。改めまして、誠にありがとうございました。\\
\end{acknowledgment}
