\begin{acknowledgment}
本論文の執筆にあたり、多くの方のご協力をいただき誠にありがとうございました。特に指導教官である片岡先生には、大変お世話になりました。毎回の報告会で的確なアドバイスをいただき、また質問にいくと親身に遅くまで相談に乗っていただき的確なアドバイスをいつもいただけたことでMPPCを用いた「低被ばく」かつ「多色」X線CTという世界でも類を見ない研究を成し遂げることができました。また、助手の有元先生には、毎度の報告会で新しい視点からアドバイスをいただき、新たな気づきを生み、実験のヒントをたくさん与えていただけました。また、デジタル・アナログLSIの設計をしていただき、本研究の今後の飛躍の最も根幹となる部分を担っていただき誠にありがとうございました。また共同研究者の森田君には、その類い稀なる洞察力、実験センスによって数々の成果を生み出し、本研究を共にできたことに心から感謝申し上げたいと思います。その優れた人格、そして研究者としての才能は日本そして世界の宝であり、今後もその才能を発揮して増々躍進することを期待したいです。また、日立金属株式会社の新田様にはシンチレータを無償でご提供いただき、本研究に大きな貢献をしていただけたことを感謝申し上げます。また、トーレック株式会社美濃部様にはX線ジェネレータの改良や、線量測定などにおいて大変お世話になりました。また、研究室の岸本さんには実験や考察に詰まった時にいつも親身に相談に乗っていただき、心より感謝申し上げます。また、同期にも大変助けられ、いつも私を支えていただけたことに心より感謝申し上げます。\\この片岡研究室で3年間、実験で悩み考え、PDCAのサイクルをひたすら繰り返した日々の経験は今後の仕事、人生において必ず役に立つと考えております。そして何より3年間、間近で見続けた片岡先生の常に新しいものを生みだし世界にイノベーションを起こし続ける姿勢は、私の今後の人生の模範となると考えております。この片岡研究室で3年過ごせたことを大変嬉しく思い、最後に改めて本論の執筆にあたりお世話になった皆様に感謝申し上げます。本当にありがとうございました。
\end{acknowledgment}
