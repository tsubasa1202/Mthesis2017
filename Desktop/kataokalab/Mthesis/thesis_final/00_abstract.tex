\begin{jabstract}
X線CT(Computed Tomography)は、医療用画像診断装置の一種で、人体を切開することなく内部の状態を立体的に観察することができる装置である。レントゲン撮影と同様にX線を用いて透過写真を得るが、レントゲン写真では三次元の被写体が二次元映像として表されるのに対し、CTでは多数の向きからX線撮影を行うことで、内部の状態を立体的に表現することが可能である。CTでしか見つからない体内の病変は大変多く、X線CTは現在の医療画像診断においても根幹をなす重要技術であるといえる。\\
しかしながらCTで必要とされる高精細画像を得るには、1mm$^2$あたり一秒間に$10^{8-9}$ctsにも及ぶ大強度のX線を人体に当てる必要があり、従って医療被ばくにおけるCTの割合が深刻化している。その被ばく量は1回で10mSvにも及ぶ場合があり、これは日本人1人当たりの平均年間自然放射線量2.1mSvに対してずっと大きい(図1.1(右))。またCT撮影は患者によって経過観察のため、年間複数回行われることは多く、その被ばく量は甚大であることが伺える。現在CTメーカー各社は、画像再構成アルゴリズム等を新たに開発することで低被ばく化を目指している。技術面においては、臨床で用いられているX線CTの多くはシンチレータとフォトダイオード(PD)を用いたエネルギー積分型CTである。すなわち、X線によって生じた電荷を一定時間蓄積して電流値を読み出すため、個々のX線パルスを分解し、エネルギー情報を取得することが出来ない。得られる画像はCT値(線減弱係数)のみをパラメータとするモノクロ画像となり、正確な物質同定が困難となる。
この問題を解決する「次世代」CTとして、複数のエネルギーバンドでデータを収集し、画像化する多色X線CT(フォトンカウンティングCT)が研究されている。一度のX線照射で様々なエネルギーバンドでのCT画像が取得可能であり、CT値が近い物質の識別やX線CT特有のアーチファクトの改善に、大きな注目を集めている。フォトンカウンティングCTの実現に向け、現在多くの医療メーカーではCdTeやCdZnTeなどの半導体を用いた直接変換型の検出器を主として研究している(たとえばPHILIPS社)。しかしながら、素子内部での電子・ホールの移動速度は遅く、臨床で求められる$10^{8-9}$ cts/mm$^2$の高計数に耐えることは非常に難しい。高計数に耐えるにはピクセルあたりの受光面積を小さくする必要があり、その結果読出しチャンネル数は膨大となる。また検出器に信号増幅機能がないためノイズに弱く、読み出しには低ノイズかつ高速応答性をもつ電荷積分アンプが不可欠となる。CdTe/CdZnTeの利用はコスト、放射線耐性の観点からも実用的でなく、既存の装置をすべて刷新する必要など、早期における実用化・臨床応用においても多くの課題を残している。\\
本研究ではMulti-Pixel Photon Counter (MPPC)と高速シンチレータを用いて、「低被ばく」かつ「多色」撮影が可能な、全く新しい革新的X線CTシステムを提案する。MPPCは約100万倍もの大きな内部増幅機能をもつ半導体光素子で、微弱信号への感度が極めて高い。この大きな内部増幅により、従来型CTより遥かに低い線量で同等以上のS/Nを実現し、一方では個々のX線パルスを弁別することで多色イメージングも可能である。本研究では1mm角のPD,APD,MPPCを用いてCT撮影を行い、低コントラスト分解能評価と空間分解能評価を行った。またMPPCを用いたK-edgeイメージングやビームハードニングの低減など、多色イメージングの効果を実証した。シンチレータは従来型CTで用いられるGd2O2S (GOS)を用い、電流を一定間隔で読み出すことで投影データを取得した。MPPCでは電流・パルスの2つの読出しを行い、パルス読出しでは時定数の短いCe:YAPを用いた。
\end{jabstract}
